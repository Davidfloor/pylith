\section{Defining the Simulation}

The parameters for PyLith are specified as a hierarchy or tree of
modules. The application assembles the hierarchy of modules from user
input and then calls the \object{main} function in the top-level
module in the same manner as a C or C++ program. The behavior of the
application is determined by the modules included in the hierarchy as
specified by the user. The Pyre framework provides the interface for
defining this hierarchy. Appendix \vref{cha:components} contains a
list of the components provided by PyLith and spatialdata.

Pyre properties correspond to simple settings in the form of strings,
integers, and real numbers. Pyre facilities correspond to software
modules. Facilities may have their own facilities (branches in the
tree) and any number of properties. See Figure
\vref{fig:Pyre:architecture} for the general concept of Pyre
facilities and properties.

\subsection{Setting PyLith Parameters}
\label{sec:setting:parameters}

There are several methods for setting input parameters for the
\filename{pylith} executable: via the command line or by using a text
file in \filename{cfg} or \filename{pml} format. Both facilities and
properties have default values provided, so you only need to set
values when you want to deviate from the default behavior.


\subsubsection{Units}

All dimensional parameters require units. The units are specified
using Python syntax, so square meters is m**2. Whitespace is not
allowed in the string, for units and dimensioned quantities are
multiplied by the units string; for example, two meters per second is
2.0*m/s. Available units are shown in Table \vref{tab:pyre:units}

\begin{table}[htbp]
\caption{Pyre supported units. Aliases are in parentheses.}
\label{tab:pyre:units}
\begin{tabular}{lp{5in}}
\textbf{Scale} & \textbf{Available Units} \\
\hline 
length & meter (m), micrometer (um, micron), millimeter (mm), centimeter (cm),
kilometer (km), inch, foot, yard, mile \\
time & second (s), nanosecond (ns), microsecond (us), millisecond (ms), minute,
hour, day, year \\
mass & kilogram (kg), gram (g), centigram (cg), milligram (mg), ounce, pound,
ton \\
pressure & pascal (Pa), kPa, MPa, GPa, bar, millibar, atmosphere (atm) \\
\hline 
\end{tabular}
\end{table}


\subsubsection{Using the Command Line}

The \commandline{-{}-help} command line argument displays links to useful
resources for learning PyLith.

Pyre uses the following syntax to change properties from the command
line. To change the value of a property of a component, use
\commandline{-{}-COMPONENT.PROPERTY=VALUE}. Each component is attached
to a facility, so the option above can also be written as
\commandline{-{}-FACILITY.PROPERTY=VALUE}.  Each facility has a
default component attached to it. A different component can be
attached to a facility by \commandline{-{}-FACILITY=NEW\_COMPONENT}.

PyLith's command-line arguments can control Pyre and PyLith properties
and facilities, MPI settings, and PETSc settings. All PyLith-related
properties are associated with the \facility{pylithapp} component. You
can get a list of all of these top-level properties along with a
description of what they do by running PyLith with the
\commandline{-{}-help-properties} command-line argument. To get
information on user-configurable facilities and components, you can
run PyLith with the \commandline{-{}-help-components} command-line
argument. To find out about the properties associated with a given
component, you can run PyLith with the
\commandline{-{}-COMPONENT.help-properties} flag:
\begin{shell}
$ pylith --problem.help-properties

# Show problem components.
$ pylith --problem.help-components

# Show bc components (bc is a component of problem).
$ pylith --problem.bc.help-components

# Show bc properties.
$ pylith --problem.bc.help-properties
\end{shell}


\subsubsection{Using a {\ttfamily .cfg} File}

Entering all those parameters via the command line involves the risk
of typographical errors. You
will generally find it easier to collect parameters into a
\filename{cfg} file. The file is composed of one or more sections
which are formatted as follows:
\begin{cfg}
<h>[pylithapp.COMPONENT1.COMPONENT2]</h>
# This is a comment.

<f>FACILITY3</f> = COMPONENT3
<p>PROPERTY1</p> = VALUE1
<p>PROPERTY2</p> = VALUE2 ; this is another comment
\end{cfg}

\usertip{We strongly recommend that you use \filename{cfg} files for your
  work.  The files are syntax-highlighted in most editors, such as
  vim, Emacs, and Atom.}


\subsubsection{Using a {\ttfamily.pml} File}

A \filename{pml} file is an XML file that specifies parameter values
in a highly structured format. It is composed of nested sections which
are formatted as follows:
\begin{lstlisting}[basicstyle=\ttfamily,frame=tb]{language=xml}
<component~name="COMPONENT1">
    <component~name="COMPONENT2">
        <property~name="PROPERTY1">VALUE1</property>
        <property~name="PROPERTY2">VALUE2</property>
    </component>
</component>
\end{lstlisting}
XML files are intended to be read and written by machines, not edited
manually by humans. The \filename{pml} file format is intended for
applications in which PyLith input files are generated by another
program, e.g., a GUI, web application, or a high-level structured
editor. This file format will not be discussed further here, but if
you are interested in using \filename{pml} files, note that \filename{pml}
files and \filename{cfg} files can be used interchangeably; in the
following discussion, a file with a \filename{pml} extension can be
substituted anywhere a \filename{cfg} file can be used.


\subsubsection{Specification and Placement of Configuration Files}

Configuration files may be specified on the command line:
\begin{shell}
$ pylith example.cfg
\end{shell}
In addition, the Pyre framework searches for configuration files named
\filename{pylithapp.cfg} in several predefined locations. You may put
settings in any or all of these locations, depending on the scope
you want the settings to have:
\begin{enumerate}
\item \filename{\$PREFIX/etc/pylithapp.cfg}, for system-wide settings;
\item \filename{\$HOME/.pyre/pylithapp/pylithapp.cfg}, for user
  settings and preferences;
\item the current directory (\filename{./pylithapp.cfg}), for local
  overrides.
\end{enumerate}

\important{The Pyre framework will search these directories for
  \filename{cfg} files matching the names of components (for example,
  \filename{timedependent.cfg}, \filename{faultcohesivekin.cfg},
  \filename{greensfns.cfg}, \filename{pointforce.cfg}, etc) and will
  attempt to assign all parameters in those files to the respective
  component.}

\important{Parameters given directly on the command line will override
  any input contained in a configuration file. Configuration files
  given on the command line override all others. The
  \filename{pylithapp.cfg} files placed in (3) will override those in
  (2), (2) overrides (1), and (1) overrides only the built-in
  defaults.}

All of the example problems are set up using configuration files and specific problems are defined by including
the appropriate configuration file on the command-line. Referring
to the directory \filename{examples/twocells/twohex8}, we have the
following.
\begin{shell}
$ ls -1 *.cfg
axialdisp.cfg
dislocation.cfg
pylithapp.cfg
sheardisp.cfg
\end{shell}
The settings in \filename{pylithapp.cfg} will be read automatically, and additional
settings are included by specifying one of the other files on the
command-line:
\begin{shell}
$ pylith axialdisp.cfg
\end{shell}
If you want to see what settings are being used, you can either examine
the \filename{cfg} files, or use the help flags as described above:
\begin{shell}
# Show components for the 'problem' facility.
$ pylith axialdisp.cfg --problem.help-components

# Show properties for the 'problem' facility.
$ pylith axialdisp.cfg --problem.help-properties

# Show components for the 'bc' facility.
$ pylith axialdisp.cfg --problem.bc.help-components

# Show properties for the 'bc' facility.
$ pylith axialdisp.cfg --problem.bc.help-properties
\end{shell}
This is generally a more useful way of determining problem settings,
since it includes default values as well as those that have been specified
in the \filename{cfg} file.


\subsubsection{List of PyLith Parameters (\filename{pylith\_info})}
\label{sec:pylithinfo}

The Python application \filename{pylith\_info} writes all of the current
parameters to a text file or JSON file (default). The default name of the JSON is \filename{pylith\_parameters.json}.
The usage synopsis is
\begin{shell}
$ pylith_info [--verbose-false] [--format={ascii,json} [--filename=pylith_parameters.json] PYLITH_ARGS
\end{shell}
where \commandline{-{}-verbose-false} turns off printing the descriptions of
the properties and components as well as the location where the
current value was set, \commandline{-{}-format=ascii} changes the
output format to a simple ASCII file, and \\
\commandline{-{}-filename=pylith\_parameters.json} sets the name of the
output file. The PyLith Parameter Viewer (see Section
\ref{sec:pylith:parameter:viewer}) provides a graphic user interface
for examining the JSON parameter file. 


% End of file
