\section{Derivation of Elasticity Equation}

For completeness we start our discussing of the governing equations
with a derivation of the elasticity equation. Consider domain $\Omega$
bounded by boundary $\Gamma$. Applying a Lagrangian description of the
conservation of momentum gives
\begin{equation}
\label{eqn:momentum:vec}
\frac{\partial}{\partial t}\int_{\Omega}\rho(\vec{x})\frac{\partial\vec{u}}{\partial t}\, d\Omega=\int_{\Omega}\vec{f}(\vec{x},t)\, d\ + \int_{\Gamma}\vec{\tau}(\vec{x},t)\, d\Gamma.
\end{equation}
The traction vector field is related to the stress tensor through
\begin{equation}
\vec{\tau}(\vec{x},t) = \tensor{\sigma}(\vec{u}) \cdot \vec{n},
\end{equation}
where $\vec{n}$ is the outward normal vector to $\Gamma$. Substituting
into equation \vref{eqn:momentum:vec} yields
\begin{equation}
\frac{\partial}{\partial t}\int_{\Omega}\rho(\vec{x})\frac{\partial\vec{u}}{\partial t}\, d\Omega=\int_{\Omega}\vec{f}(\vec{x},t)\, d\Omega+\int_{\Gamma}\tensor{\sigma}(\vec{u})\cdot\vec{n}\, d\Gamma.
\end{equation}
Applying the divergence theorem,
\begin{equation}
\int_{\Omega}\tensor{\nabla}\cdot\vec{a}\: d\Omega=\int_{\Gamma}\vec{a}\cdot\vec{n}\: d\Gamma,
\end{equation}
to the boundary integral results in
\begin{equation}
\frac{\partial}{\partial t}\int_{\Omega}\rho(\vec{x})\frac{\partial\vec{u}}{\partial t}\, d\Omega=\int_{\Omega}\vec{f}(\vec{x},t)\, d\Omega+\int_{\Omega}\tensor{\nabla}\cdot\tensor{\sigma}(\vec{u})\, d\Omega,
\end{equation}
which we can rewrite as
\begin{equation}
\int_{\Omega}\left(\rho(\vec{x})\frac{\partial^{2}\vec{u}}{\partial t^{2}}-\vec{f}(\vec{x},t)-\tensor{\nabla}\cdot\tensor{\sigma}(\vec{u})\right)\, d\Omega=\vec{0}.
\end{equation}
Because the domain $\Omega$ is arbitrary, the integrand must be the zero
vector at every location in the domain, so that we end up with
\begin{gather}
\rho(\vec{x})\frac{\partial^{2}\vec{u}}{\partial t^{2}}-\vec{f}(\vec{x},t)-\tensor{\nabla}\cdot\tensor{\sigma}=\vec{0}\text{ in }\Omega,\\
\tensor{\sigma}(\vec{u})\cdot\vec{n}=\vec{\tau}(\vec{x},t)\text{ on }\Gamma_{\tau}\text{,}\\
\vec{u}=\vec{u}_0(\vec{x},t)\text{ on }\Gamma_{u},\text{ and}\\
\vec{u}^{+}-\vec{u}^{-}=\vec{d}\text{ on }\Gamma_{f}.
\end{gather}
We specify tractions, $\vec{\tau}$, on boundary $\Gamma_{f}$, displacements,
$\vec{u^{o}}$, on boundary $\Gamma_{u}$, and slip, $\vec{d}$,
on fault interface $\Gamma_{f}$.
%(we will consider the case of fault constitutive models in Section \vref{sec:fault}).
