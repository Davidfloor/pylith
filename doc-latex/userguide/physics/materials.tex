\section{Materials (\protect\facility{materials})}
\label{sec:materials}

The material objects encapsulate the bulk behavior of the domain. This
includes both the governing equations and the associated bulk rheology.

\subsection{Specifying Material Properties}

Associating material properties with a given cell involves several
steps. 
\begin{enumerate}
\item In the mesh generation process assign a material identifier to each
cell.
\item Define material property groups corresponding to each material
  identifier. In CUBIT/Trelis this is done by creating the blocks as
  part of the boundary conditions.
\item Provide the settings for each material group in the parameters,
  i.e., \filename{cfg} file.
\item Specify the parameters for the material properties, e.g., linear
  variation in density with depth, using a spatial database file. This
  allows variation of the material properties across cells with the
  same material identifier.
\end{enumerate}

Facilities and properties common to all materials:
\begin{inventory}
\propertyitem{id}{This is the material identifier that matches the integer value
assigned to each cell in the mesh generation process (default=0);}
\propertyitem{label}{Name or label for the material (default=""), this is used in error and
diagnostic reports);}
\facilityitem{db\_auxiliary\_field}{Spatial database for physical property parameters;}
\facilityitem{observers}{Observers of physics, i.e., output (default=[\object{PhysicsObserver}]); and}
\facilityitem{auxiliary\_subfields}{Discretization information for auxiliary subfields.}
\end{inventory}


\subsection{Elasticity (\protect\object{Elasticity})}

The \object{Elasticity} is used to solve the elasticity equation with
or without inertia. Whether inertia or body forces are included is
determined by the \object{Elasticity} property settings. Gravitational
body forces are included if the \facility{gravity\_field} is set in
the \object{Problem}. The properties and facilities are;
\begin{inventory}
 \facilityitem{derived\_subfields}{Discretization information for derived subfields.}
 \propertyitem{use\_inertia}{Include inertial term in elasticity
    equation (default=False);}
  \propertyitem{use\_body\_force}{Include body force term in
    elasticity equation (default=False); and}
  \facilityitem{bulk\_rheology}{Bulk rheology for an elastic
    material.}
\end{inventory}
Table \vref{tab:elasticity:rheologies} lists the bulk rheologies
implemented for the elaticity equation.

\begin{table}[htbp]
  \caption{Elasticity bulk rheologies.}
  \label{tab:elasticity:rheologies}
  \begin{tabular}{ll}
    \toprule
    \thead{Bulk Rheology} & \thead{Description} \\
    \midrule
      \object{IsotropicLinearElasticity} & Isotropic, linear elasticity \\
      \object{IsotropicLinearMaxwell} & Isotropic, linear Maxwell viscoelasticity \\
      \object{IsotropicLinearGenMaxwell} & Isotropic, generalized Maxwell viscoelasticity \\
      \object{IsotropicPowerLaw} & Isotropic, power-law viscoelasticity \\
      \object{IsotropicDruckerPrager} & Isotropic, Drucker-Prager elastoplasticity \\
    \bottomrule
  \end{tabular}
\end{table}

\userwarning{The \object{IsotropicDruckerPrager} rheology has not been
  implemented in this beta release.}

\begin{table}[htbp]
  \caption{Auxiliary subfields for elasticity bulk rheologies.}
  \label{tab:elasticity:auxiliary:subfields}
  \begin{tabular}{lcccccl}
    \toprule
    \multirow{2}{*}{\thead{Subfield}} & \multicolumn{5}{c}{\thead{Bulk Rheologies}} & \multirow{2}{*}{\thead{Components}} \\
                              & \thead{L} & \thead{LM} & \thead{GM} & \thead{PL} & \thead{DP} & \\
    \midrule
    density & X & X & X & X & X & \textemdash \\
    vp (P-wave speed) & X & X & X & X & X & \textemdash\\
    vs (S-wave speed) & X & X & X & X & X & \textemdash\\
    body\_force & O & O & O & O & O & x, y, z \\
    gravitational\_acceleration & O & O & O & O & O & x, y, z \\
    shear\_modulus & I & I & I & I & I & \textemdash \\
    bulk\_modulus & I & I & I & I & I & \textemdash \\
    reference\_stress & O & O & O & O & O & xx, yy, zz, xy, yz, xz \\
    reference\_strain & O & O & O & O & O & xx, yy, zz, xy, yz, xz \\
    maxwell\_time & & I & & & & \textemdash \\
    viscosity & & X & & & & \textemdash \\
    viscosity\_1 &&& X &&& \textemdash \\
    viscosity\_2 &&& X &&& \textemdash \\
    viscosity\_3 &&& X &&& \textemdash \\
    shear\_ratio\_1 &&& X &&& \textemdash \\
    shear\_ratio\_2 &&& X &&& \textemdash \\
    shear\_ratio\_3 &&& X &&& \textemdash \\
    total\_strain & & X & & X & & xx, yy, zz, xy, yz, xz \\
    viscous\_strain & & X & & X & & xx, yy, zz, xy, yz, xz \\
    viscous\_strain\_1 &&& X &&& xx, yy, zz, xy, yz, xz \\
    viscous\_strain\_2 &&& X &&& xx, yy, zz, xy, yz, xz \\
    viscous\_strain\_3 &&& X &&& xx, yy, zz, xy, yz, xz \\
    power\_law\_exponent &&&& X && \textemdash \\
    reference\_strain\_rate &&&& X && \textemdash \\
    reference\_stress &&&& X && \textemdash \\
    cohesion &&&&& X & \textemdash \\
    friction\_angle &&&&& X & \textemdash \\
    dilatation\_angle &&&&& X & \textemdash \\
    alpha\_yield &&&&& I & \textemdash \\
    alpha\_flow &&&&& I & \textemdash \\
    beta &&&&& I & \textemdash \\
    plastic\_strain &&&&& X & xx, yy, zz, xy, yz, xz \\
    \bottomrule
  \end{tabular} \\
  X: required value in \facility(db\_auxiliary\_fields) spatial database\\
  O: optional value in \facility(db\_auxiliary\_fields) spatial database\\
  I: internal; computed from inputs\\
  L: isotropic, linear elasticity\\
  ML: isotropic linear Maxwell viscoelasticity\\
  GM: isotropic generalized linear Maxwell viscoelasticity\\
  PL: isotropic power-law viscoelasticity\\
  DP: isotropic Drucker-Prager elastoplasticity
\end{table}

\begin{table}[htbp]
  \caption{Derived subfields for elasticity bulk rheologies.}
  \label{tab:elasticity:derived:subfields}
  \begin{tabular}{lcccccl}
    \toprule
    \multirow{2}{*}{\thead{Subfield}} & \multicolumn{5}{c}{\thead{Bulk Rheologies}} & \multirow{2}{*}{\thead{Components}} \\
                              & \thead{L} & \thead{LM} & \thead{GM} & \thead{PL} & \thead{DP} & \\
    \midrule
    cauchy\_stress & \yes & \yes & \yes & \yes & \yes & xx, yy, zz, xy, yz, xz \\
    cauchy\_strain & \yes & \yes & \yes & \yes & \yes & xx, yy, zz, yz, yz, xz \\
    \bottomrule
  \end{tabular}
\end{table}


Property common to all bulk rheologies for elasticity:
\begin{inventory}
  \propertyitem{use\_reference\_state}{Flag indicating to compute
    deformation relative to the supplied reference state
    (default=False).}
\end{inventory}

\begin{cfg}[Parameters for two materials in a \filename{cfg} file]
<h>[pylithapp.problem]</h>
<p>materials</p> = [elastic, viscoelastic]
<f>materials.viscoelastic.bulk_rheology</f> = pylith.materials.IsotropicLinearMaxwell

<h>[pylithapp.problem.materials.elastic]</h>
<p>label</p> = Elastic material
<p>id</p> = 1
<f>db_auxiliary_field</f> = spatialdata.spatialdb.UniformDB
<p>db_auxiliary_field.label</p> = Elastic properties
<p>db_auxiliary_field.values</p> = [density, vs, vp]
<p>db_auxiliary_field.data</p> = [2500*kg/m**3, 3.0*km/s, 5.2915026*km/s]

<p>observers.observer.writer.filename</p> = output/step01-elastic.h5

# Set the discretization of the material auxiliary fields (properties).
# With uniform material properties, we can use a basis order of 0.
<p>auxiliary_subfields.density.basis_order</p> = 0
<p>auxiliary_subfields.density.quadrature_order</p> = 1

<h>[pylithapp.problem.materials.elastic.bulk_rheology]</h>
<p>auxiliary_subfields.bulk_modulus.basis_order</p> = 0
<p>auxiliary_subfields.bulk_modulus.quadrature_order</p> = 1

<p>auxiliary_subfields.shear_modulus.basis_order</p> = 0
<p>auxiliary_subfields.shear_modulus.quadrature_order</p> = 1


<h>[pylithapp.problem.materials.viscoelastic]</h>
<p>label</p> = Viscoelastic material
<p>id</p> = 2
<f>db_auxiliary_field</f> = spatialdata.spatialdb.SimpleGridDB
<p>db_auxiliary_field.label</p> = Viscoelastic properties
<p>db_auxiliary_field.filename</p> = mat_viscoelastic.spatialdb

<p>observers.observer.writer.filename</p> = output/step01-viscoelastic.h5

# Set the discretization of the material auxiliary fields (properties).
# We will assume we have a linear variation in material properties in
# mat_viscoelastic.spatialdb, so we use a basis order of 1.
<p>auxiliary_subfields.density.basis_order</p> = 1
<p>auxiliary_subfields.density.quadrature_order</p> = 1

<h>[pylithapp.problem.materials.elastic.bulk_rheology]</h>
<p>auxiliary_subfields.bulk_modulus.basis_order</p> = 1
<p>auxiliary_subfields.bulk_modulus.quadrature_order</p> = 1

<p>auxiliary_subfields.shear_modulus.basis_order</p> = 1
<p>auxiliary_subfields.shear_modulus.quadrature_order</p> = 1

<p>auxiliary_subfields.viscosity.basis_order</p> = 1
<p>auxiliary_subfields.viscosity.quadrature_order</p> = 1
\end{cfg}


\subsection{Incompressible Elasticity (\protect\object{IncompressibleElasticity})}

Estimating realistic distributions of initial stress fields consistent
with gravitational body forces can be quite difficult due to our lack
of knowledge of the deformation history. A simple way to approximate
the lithostatic load is to solve for the stress field imposed by
gravitational body forces assuming an incompressible elastic
material. This limits the volumetric deformation. In this context we
do not include inertia, so the \object{IncompressibleElasticity}
object does not include an inertial term. Gravitational
body forces are included if the \facility{gravity\_field} is set in
the \object{Problem}. The properties and facilities are;
\begin{inventory}
 \facilityitem{derived\_subfields}{Discretization information for derived subfields.}
  \propertyitem{use\_body\_force}{Include body force term in
    elasticity equation (default=False); and}
  \facilityitem{bulk\_rheology}{Bulk rheology for an elastic
    material.}
\end{inventory}
Table \vref{tab:incompressible:elasticity:rheologies} lists the bulk rheologies
implemented for the elaticity equation.

\begin{table}[htbp]
  \caption{Incompressible elasticity bulk rheologies.}
  \label{tab:incompressible:elasticity:rheologies}
  \begin{tabular}{ll}
    \toprule
    \thead{Bulk Rheology} & \thead{Description} \\
    \midrule
    \object{IsotropicLinearIncompElasticity} & Isotropic, linear incompressible elasticity \\
    \bottomrule
  \end{tabular}
\end{table}

\begin{table}[htbp]
  \caption{Auxiliary subfields for incompressible elasticity bulk rheologies.}
  \label{tab:incompressible:elasticity:auxiliary:subfields}
  \begin{tabular}{lccccl}
    \toprule
    \multirow{2}{*}{\thead{Subfield}} & \multicolumn{4}{c}{\thead{Bulk Rheologies}} & \multirow{2}{*}{\thead{Components}} \\
                              & \thead{L} & \thead{LM} & \thead{GM} & \thead{PL} & \\
    \midrule
    density & X & X & X & X & \textemdash \\
    vp (P-wave speed) & X & X & X & X & \textemdash\\
    vs (S-wave speed) & X & X & X & X & \textemdash\\
    body\_force & O & O & O & O & x, y, z \\
    gravitational\_acceleration & O & O & O & O & x, y, z \\
    shear\_modulus & I & I & I & I & \textemdash \\
    bulk\_modulus & I & I & I & I & \textemdash \\
    reference\_stress & O & O & O & O & xx, yy, zz, xy, yz, xz \\
    reference\_strain & O & O & O & O & xx, yy, zz, xy, yz, xz \\
    \bottomrule
  \end{tabular} \\
  X: required value in \facility(db\_auxiliary\_fields) spatial database\\
  O: optional value in \facility(db\_auxiliary\_fields) spatial database\\
  I: internal; computed from inputs\\
  L: isotropic, linear elasticity\\
  ML: isotropic linear Maxwell viscoelasticity\\
  GM: isotropic generalized linear Maxwell viscoelasticity\\
  PL: isotropic power-law viscoelasticity\\
\end{table}

\begin{table}[htbp]
  \caption{Derived subfields for incompressible elasticity bulk rheologies.}
  \label{tab:incompressible:elasticity:derived:subfields}
  \begin{tabular}{lccccl}
    \toprule
    \multirow{2}{*}{\thead{Subfield}} & \multicolumn{4}{c}{\thead{Bulk Rheologies}} & \multirow{2}{*}{\thead{Components}} \\
                              & \thead{L} & \thead{LM} & \thead{GM} & \thead{PL} & \\
    \midrule
    cauchy\_stress & \yes & \yes & \yes & \yes & xx, yy, zz, xy, yz, xz \\
    cauchy\_strain & \yes & \yes & \yes & \yes & xx, yy, zz, yz, yz, xz \\
    \bottomrule
  \end{tabular}
\end{table}


Property common to all bulk rheologies for elasticity:
\begin{inventory}
  \propertyitem{use\_reference\_state}{Flag indicating to compute
    deformation relative to the supplied reference state
    (default=False).}
\end{inventory}

\begin{cfg}[Parameters for an incompressible material in a \filename{cfg} file]
<h>[pylithapp.problem]</h>
<p>materials</p> = [elastic]
<f>materials.elastic</f> = pylith.materials.IncompressibleElasticity
# Use the default bulk_rheology: IsotropicLinearIncompElasticity

<f>gravity_field</f> = spatialdata.spatialdb.GravityField
<p>gravity_field.gravity_dir</p> = [0.0, -1.0, 0.0]

# With incompressible elasticity, the solution subfields are displacement and pressure.
<f>solution</f> = pylith.problems.SolnDispPres

<h>[pylithapp.problem.materials.elastic]</h>
<p>label</p> = Elastic material
<p>id</p> = 1
<f>db_auxiliary_field</f> = spatialdata.spatialdb.UniformDB
<p>db_auxiliary_field.label</p> = Elastic properties
<p>db_auxiliary_field.values</p> = [density, vs, vp]
<p>db_auxiliary_field.data</p> = [2500*kg/m**3, 3.0*km/s, 1.0e+12*km/s]

<p>observers.observer.writer.filename</p> = output/step01-elastic.h5

# Set the discretization of the material auxiliary fields (properties).
# With uniform material properties, we can use a basis order of 0.
<p>auxiliary_subfields.density.basis_order</p> = 0
<p>auxiliary_subfields.density.quadrature_order</p> = 1

<h>[pylithapp.problem.materials.elastic.bulk_rheology]</h>
<p>auxiliary_subfields.bulk_modulus.basis_order</p> = 0
<p>auxiliary_subfields.bulk_modulus.quadrature_order</p> = 1

<p>auxiliary_subfields.shear_modulus.basis_order</p> = 0
<p>auxiliary_subfields.shear_modulus.quadrature_order</p> = 1
\end{cfg}


% \subsection{Cauchy Stress Tensor and Second Piola-Kirchoff Stress Tensor}

% In outputting the stress tensor (see Tables
% \vref{tab:materials:output} and \vref{tab:materials:statevars}),
% the tensor used internally in the formulation of the governing
% equation is the \texttt{stress} field available for output. For the
% infinitesimal strain formulation this is the Cauchy stress tensor; for
% the finite strain formulation, this is the second Piola-Kirchoff
% stress tensor. The user may also explicitly request output of the
% Cauchy stress tensor (\texttt{cauchy\_stress} field). Obviously, this
% is identical to the \texttt{stress} field when using the infinitesimal
% strain formulation.  See section \vref{sec:small:strain:formulation}
% for a discussion of the relationship between the Cauchy stress tensor
% and the second Piola-Kirchhoff stress tensor.

% \important{Although the second Piola-Kirchoff stress tensor has little
%   physical meaning, the second Piola-Kirchoff stress tensor (not the
%   Cauchy stress tensor) values should be specified in the initial
%   stress database when using the finite strain formulation.}

% \begin{table}[htbp]
%   \caption{Values in spatial databases for the elastic material
%     constitutive models.}
% \begin{tabular}{lll}
% \textbf{Spatial database} & \textbf{Value} & \textbf{Description}\\
% \hline 
% \facility{db\_properties} & \texttt{vp} & Compressional wave speed, $v_{p}$\\
%  & \texttt{vs} & Shear wave speed, $v_{s}$\\
%  & \texttt{density} & Density, $\rho$\\
% \facility{db\_initial\_stress} & \texttt{stress-xx}, \ldots & Initial stress components\\
% \facility{db\_initial\_strain} & \texttt{total-strain-xx}, \ldots & Initial strain components\\
% \hline 
% \end{tabular}
% \end{table}


% End of file
